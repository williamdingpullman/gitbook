\documentclass[]{book}
\usepackage{lmodern}
\usepackage{amssymb,amsmath}
\usepackage{ifxetex,ifluatex}
\usepackage{fixltx2e} % provides \textsubscript
\ifnum 0\ifxetex 1\fi\ifluatex 1\fi=0 % if pdftex
  \usepackage[T1]{fontenc}
  \usepackage[utf8]{inputenc}
\else % if luatex or xelatex
  \ifxetex
    \usepackage{mathspec}
  \else
    \usepackage{fontspec}
  \fi
  \defaultfontfeatures{Ligatures=TeX,Scale=MatchLowercase}
\fi
% use upquote if available, for straight quotes in verbatim environments
\IfFileExists{upquote.sty}{\usepackage{upquote}}{}
% use microtype if available
\IfFileExists{microtype.sty}{%
\usepackage[]{microtype}
\UseMicrotypeSet[protrusion]{basicmath} % disable protrusion for tt fonts
}{}
\PassOptionsToPackage{hyphens}{url} % url is loaded by hyperref
\usepackage[unicode=true]{hyperref}
\hypersetup{
            pdftitle={GLMM, Concepts, \& R},
            pdfauthor={Bill Last Updated:},
            pdfborder={0 0 0},
            breaklinks=true}
\urlstyle{same}  % don't use monospace font for urls
\usepackage{natbib}
\bibliographystyle{apalike}
\usepackage{color}
\usepackage{fancyvrb}
\newcommand{\VerbBar}{|}
\newcommand{\VERB}{\Verb[commandchars=\\\{\}]}
\DefineVerbatimEnvironment{Highlighting}{Verbatim}{commandchars=\\\{\}}
% Add ',fontsize=\small' for more characters per line
\usepackage{framed}
\definecolor{shadecolor}{RGB}{248,248,248}
\newenvironment{Shaded}{\begin{snugshade}}{\end{snugshade}}
\newcommand{\KeywordTok}[1]{\textcolor[rgb]{0.13,0.29,0.53}{\textbf{#1}}}
\newcommand{\DataTypeTok}[1]{\textcolor[rgb]{0.13,0.29,0.53}{#1}}
\newcommand{\DecValTok}[1]{\textcolor[rgb]{0.00,0.00,0.81}{#1}}
\newcommand{\BaseNTok}[1]{\textcolor[rgb]{0.00,0.00,0.81}{#1}}
\newcommand{\FloatTok}[1]{\textcolor[rgb]{0.00,0.00,0.81}{#1}}
\newcommand{\ConstantTok}[1]{\textcolor[rgb]{0.00,0.00,0.00}{#1}}
\newcommand{\CharTok}[1]{\textcolor[rgb]{0.31,0.60,0.02}{#1}}
\newcommand{\SpecialCharTok}[1]{\textcolor[rgb]{0.00,0.00,0.00}{#1}}
\newcommand{\StringTok}[1]{\textcolor[rgb]{0.31,0.60,0.02}{#1}}
\newcommand{\VerbatimStringTok}[1]{\textcolor[rgb]{0.31,0.60,0.02}{#1}}
\newcommand{\SpecialStringTok}[1]{\textcolor[rgb]{0.31,0.60,0.02}{#1}}
\newcommand{\ImportTok}[1]{#1}
\newcommand{\CommentTok}[1]{\textcolor[rgb]{0.56,0.35,0.01}{\textit{#1}}}
\newcommand{\DocumentationTok}[1]{\textcolor[rgb]{0.56,0.35,0.01}{\textbf{\textit{#1}}}}
\newcommand{\AnnotationTok}[1]{\textcolor[rgb]{0.56,0.35,0.01}{\textbf{\textit{#1}}}}
\newcommand{\CommentVarTok}[1]{\textcolor[rgb]{0.56,0.35,0.01}{\textbf{\textit{#1}}}}
\newcommand{\OtherTok}[1]{\textcolor[rgb]{0.56,0.35,0.01}{#1}}
\newcommand{\FunctionTok}[1]{\textcolor[rgb]{0.00,0.00,0.00}{#1}}
\newcommand{\VariableTok}[1]{\textcolor[rgb]{0.00,0.00,0.00}{#1}}
\newcommand{\ControlFlowTok}[1]{\textcolor[rgb]{0.13,0.29,0.53}{\textbf{#1}}}
\newcommand{\OperatorTok}[1]{\textcolor[rgb]{0.81,0.36,0.00}{\textbf{#1}}}
\newcommand{\BuiltInTok}[1]{#1}
\newcommand{\ExtensionTok}[1]{#1}
\newcommand{\PreprocessorTok}[1]{\textcolor[rgb]{0.56,0.35,0.01}{\textit{#1}}}
\newcommand{\AttributeTok}[1]{\textcolor[rgb]{0.77,0.63,0.00}{#1}}
\newcommand{\RegionMarkerTok}[1]{#1}
\newcommand{\InformationTok}[1]{\textcolor[rgb]{0.56,0.35,0.01}{\textbf{\textit{#1}}}}
\newcommand{\WarningTok}[1]{\textcolor[rgb]{0.56,0.35,0.01}{\textbf{\textit{#1}}}}
\newcommand{\AlertTok}[1]{\textcolor[rgb]{0.94,0.16,0.16}{#1}}
\newcommand{\ErrorTok}[1]{\textcolor[rgb]{0.64,0.00,0.00}{\textbf{#1}}}
\newcommand{\NormalTok}[1]{#1}
\usepackage{longtable,booktabs}
% Fix footnotes in tables (requires footnote package)
\IfFileExists{footnote.sty}{\usepackage{footnote}\makesavenoteenv{long table}}{}
\usepackage{graphicx,grffile}
\makeatletter
\def\maxwidth{\ifdim\Gin@nat@width>\linewidth\linewidth\else\Gin@nat@width\fi}
\def\maxheight{\ifdim\Gin@nat@height>\textheight\textheight\else\Gin@nat@height\fi}
\makeatother
% Scale images if necessary, so that they will not overflow the page
% margins by default, and it is still possible to overwrite the defaults
% using explicit options in \includegraphics[width, height, ...]{}
\setkeys{Gin}{width=\maxwidth,height=\maxheight,keepaspectratio}
\IfFileExists{parskip.sty}{%
\usepackage{parskip}
}{% else
\setlength{\parindent}{0pt}
\setlength{\parskip}{6pt plus 2pt minus 1pt}
}
\setlength{\emergencystretch}{3em}  % prevent overfull lines
\providecommand{\tightlist}{%
  \setlength{\itemsep}{0pt}\setlength{\parskip}{0pt}}
\setcounter{secnumdepth}{5}
% Redefines (sub)paragraphs to behave more like sections
\ifx\paragraph\undefined\else
\let\oldparagraph\paragraph
\renewcommand{\paragraph}[1]{\oldparagraph{#1}\mbox{}}
\fi
\ifx\subparagraph\undefined\else
\let\oldsubparagraph\subparagraph
\renewcommand{\subparagraph}[1]{\oldsubparagraph{#1}\mbox{}}
\fi

% set default figure placement to htbp
\makeatletter
\def\fps@figure{htbp}
\makeatother

\usepackage{booktabs}
\usepackage{amsthm}
\makeatletter
\def\thm@space@setup{%
  \thm@preskip=8pt plus 2pt minus 4pt
  \thm@postskip=\thm@preskip
}
\makeatother

\title{GLMM, Concepts, \& R}
\author{Bill Last Updated:}
\date{19 January, 2020}

\begin{document}
\maketitle

{
\setcounter{tocdepth}{1}
\tableofcontents
}
\chapter*{Preface: Motivation}\label{my-section}
\addcontentsline{toc}{chapter}{Preface: Motivation}

All the notes I have done here are the preparation for my stat master
project, which will be about Generalized Linear Mixed Models. While I
have tried my best, probably there are still some typos and errors.
Please feel free to let me know in case you find one. Thank you!

\chapter{LM and GLM}\label{lm-and-glm}

Before moving to LMM, I would like to review LM and GLM first.

\section{LM}\label{lm}

\[Y|X \sim N(\mu(X),\sigma^2 I)\]

\[E(Y|X)=\mu(X)=X^T \beta\]

where,

\(\mu(X): random component\)

\(X^T \beta: covariates\)

\section{GLM-Definition}\label{glm-definition}

Ref:
\url{https://ocw.mit.edu/courses/mathematics/18-650-statistics-for-applications-fall-2016/lecture-slides/MIT18_650F16_GLM.pdf}

\[Y \sim exponential family\] Link function

\[g(\mu(X))=X^T \beta\]

\section{GLM-log link example}\label{glm-log-link-example}

\[\mu_i = \gamma e^{\delta t_i}\] Link function is log link, and it
becomes:

\[log(\mu_i) = log(\gamma) + log(\delta t_i)=\beta_0+\beta_1 t_i\] (This
is somehow similar to Poisson distribution.)

\section{GLM-Reciprocal link:}\label{glm-reciprocal-link}

\[\mu_i=\frac{\alpha x_i}{h+x_i}\]

Reciprocal link:

\[g(\mu_i)=\frac{1}{\mu_i}=\frac{1}{\alpha}+\frac{h}{\alpha}\frac{1}{x_i}=\beta_0+\beta_1 \frac{1}{x_i}\]

\section{GLM-exponential family:}\label{glm-exponential-family}

In a more general sense, for exponential family:

\[\begin{aligned} P_{\theta}(X)=P(X, \theta)&= e^{\sum \eta_i(\theta)T_i(X)} C(\theta)h(x)\\ &=e^{\sum \eta_i(\theta)T_i(X)} e^{-log(\frac{1}{c(\theta)})}h(x) \\ &= e^{\sum \eta_i(\theta)T_i(X)-log(\frac{1}{c(\theta)})} h(x)  \\&= e^{\sum \eta_i(\theta)T_i(X)-B(\theta)} h(x) \end{aligned}\]
\textbf{Normal distribution}

For normal distributions, it belongs to exponential family.

\[\begin{aligned} P_{\theta}(X) &= \frac{1}{\sigma\sqrt{2\pi}} e^{-\frac{1}{2}\frac{(x-\mu)^2}{\sigma^2}}\\ &=e^{-\frac{1}{2}\frac{(x-\mu)^2}{\sigma^2}} e^{log(\frac{1}{\sigma\sqrt{2\pi}})} \\ &= e^{-\frac{1}{2}\frac{(x-\mu)^2}{\sigma^2}-log (\sigma\sqrt{2\pi})} \\ &= e^{-\frac{1}{2\sigma^2}x^2-\frac{1}{2\sigma^2} \mu^2+\frac{x\mu}{\sigma^2}-log(\sqrt{2\pi}\sigma)}\\ &=e^{-\frac{1}{2\sigma^2}x^2+\frac{x\mu}{\sigma^2}-(\frac{1}{2\sigma^2} \mu^2+log(\sqrt{2\pi}\sigma))} \end{aligned}\]
Where,

\(\eta_1 =-\frac{1}{2\sigma^2}\) and \(T_1(x)=x^2\)

\(\eta_2 =-\frac{\mu}{\sigma^2}\) and \(T_2(x)=x\)

\(B(\theta)=\frac{1}{2\sigma^2} \mu^2+log(\sqrt{2\pi}\sigma)\)

\(h(x)=1\)

In the case above, \(\theta=(\mu, \sigma^2)\). If \(\sigma^2\) is known,
\(\theta=\mu\). In this case, we can rewrite the normal pdf as follows.

\[\begin{aligned} P_{\theta}(X) &=e^{-\frac{1}{2\sigma^2}x^2-\frac{1}{2\sigma^2} \mu^2+\frac{x\mu}{\sigma^2}-log(\sqrt{2\pi}\sigma)}\\ &=e^{\frac{x\mu}{\sigma^2}-\frac{1}{2\sigma^2} \mu^2}e^{-\frac{1}{2\sigma^2}x^2-log(\sqrt{2\pi}\sigma)} \end{aligned}\]
Where,

\(\eta_1 =-\frac{\mu}{\sigma^2}\) and \(T_1(x)=x\)

\(B(\theta)=\frac{1}{2\sigma^2} \mu^2\)

\(\begin{aligned} h(x) &=e^{-\frac{1}{2\sigma^2}x^2-log(\sqrt{2\pi}\sigma)} \\&=\frac{1}{\sigma\sqrt{2\pi}}e^{-\frac{1}{2}\frac{x^2}{\sigma^2}} \end{aligned}\)

Thus, we can see that \(h(x)\) is a normal pdf \(\sim N(0, \sigma^2)\).

\textbf{Bernoulli}

Another example, \(x\) is descrete. For example, Bernoulli:

\[ \begin{aligned} &= p^x(1-p)^{1-x}  \\ &=e^{log(p^x(1-p)^{1-x})} \\ &= e^{xlog(p)+(1-x)log(1-p)}\\ &= e^{xlog(p)-xlog(1-p)+log(1-p)}\\ &=e^{xlog(\frac{p}{1-p})+log(1-p)} \end{aligned}\]

Where,

\(\eta_1 =log(\frac{p}{1-p})\) and \(T_1(x)=x\)

\(B(\theta)=log(\frac{1}{1-p})\)

\(h(x) =1\)

\section{Canonical exponential
family}\label{canonical-exponential-family}

Canonical exponential family:

\[f_{\theta}(x)=e^{\frac{x\theta-b(\theta)}{\phi}+c(x,\phi)}\] where,
\(b(.)\) and \(c(.,.)\) are known.

\textbf{Normal distribution}

Again, use the normal pdf:

\[\begin{aligned} P_{\theta}(X) &= \frac{1}{\sigma\sqrt{2\pi}} e^{-\frac{1}{2}\frac{(x-\mu)^2}{\sigma^2}}\\ &=e^{-\frac{1}{2}\frac{(x-\mu)^2}{\sigma^2}} e^{log(\frac{1}{\sigma\sqrt{2\pi}})} \\ &= e^{-\frac{1}{2}\frac{(x-\mu)^2}{\sigma^2}-log (\sigma\sqrt{2\pi})} \\ &= e^{-\frac{1}{2\sigma^2}x^2-\frac{1}{2\sigma^2} \mu^2+\frac{x\mu}{\sigma^2}-log(\sqrt{2\pi}\sigma)}\\ &= e^{\frac{x\mu}{\sigma^2}-\frac{\mu^2}{2\sigma^2}+(-\frac{1}{2\sigma^2}x^2-log(\sqrt{2\pi}\sigma)) } \\ &=e^{\frac{x\mu-\frac{1}{2}\mu^2}{\sigma^2}+(-\frac{1}{2\sigma^2}x^2-log(\sqrt{2\pi}\sigma)) } \end{aligned}\]

Where (we assume \(\sigma^2\) is known.),

\(\theta=\mu\)

\(\phi =\sigma^2\)

\(b(\theta)=\frac{1}{2}\theta^2\)

\(\begin{aligned} c(x, \phi) &=-\frac{1}{2\sigma^2}x^2-log(\sqrt{2\pi}\sigma) \\ &=-\frac{1}{2\sigma^2}x^2-\frac{1}{2}log(2\pi\sigma^2) \\ &=-\frac{1}{2}(\frac{x^2}{\sigma^2}+log(2\pi\sigma^2)) \\ &=-\frac{1}{2}(\frac{x^2}{\phi}+log(2\pi \phi)) \end{aligned}\)

\section{Canonical exponential family - Expected value and
variance}\label{canonical-exponential-family---expected-value-and-variance}

\textbf{First derivative}

Canonical exponential family:

\[f_{\theta}(x)=e^{\frac{x\theta-b(\theta)}{\phi}+c(x,\phi)}\]

log likelihood (only one observation) \[log f_{\theta}(x)\]

\[\begin{aligned} E[\frac{\partial (logf_{\theta}(X))}{\partial \theta} ] &=E[\frac{\frac{\partial f_{\theta}(X)}{\partial \theta}}{f_{\theta}(X)}] \\ &= \int \frac{\frac{\partial f_{\theta}(X)}{\partial \theta}}{f_{\theta}(X)} f_{\theta}(X) dx  \\ &= \int \frac{\partial f_{\theta}(X)}{\partial \theta} dx \\ &= \frac{\partial}{\partial \theta} \int f_{\theta}(X)dx \\ &= \frac{\partial 1}{\partial \theta} \\ &=0  \end{aligned}\]
\textbf{Second derivative}

Second derivative

\[\begin{aligned} E[\frac{\partial^2 (logf_{\theta}(X))}{\partial \theta^2} ] &=E[ \frac{\partial}{\partial \theta}(\frac{\frac{\partial f_{\theta}(X)}{\partial \theta}}{f_{\theta}(X)})] \\  
&=E[\frac{\frac{\partial^2 f_{\theta}(X)}{\partial \theta^2}f_{\theta}(X)-(\frac{\partial f_{\theta}(X)}{\partial \theta})^2}{f^2_{\theta}(X)}] \\ &= \int \frac{\frac{\partial^2 f_{\theta}(X)}{\partial \theta^2}f_{\theta}(X)-(\frac{\partial f_{\theta}(X)}{\partial \theta})^2}{f_{\theta}(X)}dx \\ &=\int (\frac{\partial^2 f_{\theta}(X)}{\partial \theta^2} - \frac{(\frac{\partial f_{\theta}(X)}{\partial \theta})^2}{f_{\theta}(X)})dx  \\ 
&= \int \frac{\partial^2 f_{\theta}(X)}{\partial \theta^2} dx -\int \frac{(\frac{\partial f_{\theta}(X)}{\partial \theta})^2}{f_{\theta}(X)}dx \\ 
&=\frac{\partial^2}{\partial \theta^2}\int f_{\theta}(X) dx -\int \frac{(\frac{\partial f_{\theta}(X)}{\partial \theta})^2}{f_{\theta}(X)}dx \\ 
&=0-\int \frac{(\frac{\partial f_{\theta}(X)}{\partial \theta})^2}{f_{\theta}(X)}dx \\
&=0-\int \frac{(\frac{\partial f_{\theta}(X)}{\partial \theta})^2}{(f_{\theta}(X))^2}f_{\theta}(x)dx \\ &= - E[(\frac{\frac{\partial f_{\theta}(X)}{\partial \theta}}{f_{\theta}(X)})^2]\\ &= -E[(\frac{\partial (logf_{\theta}(X))}{\partial \theta})^2] \end{aligned}\]

Based on the first derivative, we can get:

\[log(f_{\theta}(X))=\frac{X\theta-b(\theta)}{\phi}+c(X,\phi)\]
\[E[\frac{\partial (log(f_{\theta}(X)))}{\partial \theta}]= E[\frac{X-b^{'}(\theta)}{\phi}]=\frac{E(X)-b^{'}(\theta)}{\phi}=0\]
Thus, we can get,

\[E(X)=b^{'}(\theta)\]

For second derivative, from the calculation above, we know that,

\[\begin{aligned} E[\frac{\partial^2 (logf_{\theta}(X))}{\partial \theta^2}]&=-E[(\frac{\partial (logf_{\theta}(X))}{\partial \theta})^2] \\ &= -E[(\frac{X-b^{'}(\theta)}{\phi})^2]\\ &=-E[(\frac{X-E(X)}{\phi})^2] \\ &= -\frac{Var(X)}{\phi^2}\end{aligned}\]

At the same time,
\[\begin{aligned} E[\frac{\partial^2 (logf_{\theta}(X))}{\partial \theta^2}]&=E[\frac{\partial (\frac{X-b^{'}(\theta)}{\phi})}{\partial \theta}] \\ &= E[-\frac{b^{''}(\theta)}{\phi}]\\ &= - \frac{b^{''}(\theta)}{\phi} \end{aligned}\]
Thus,

\[Var(X)=b^{''}(\theta) \phi\]

\section{Expected value and variance - Possion
Example}\label{expected-value-and-variance---possion-example}

Example of possion distribution

\[P(\lambda)=\frac{\lambda^k e^{-\lambda}}{k!}\]

If we put \(k\) as \(y\), and \(\lambda\) as \(\mu\), we can get:

\[P(\mu)=\frac{\mu^y e^{-\mu}}{y!}\] Compare to,

\[f_{\theta}(y)=e^{\frac{y\theta-b(\theta)}{\phi}+c(y,\phi)}\]

We can write it as the exponential format:

\[\begin{aligned} P(\mu) &=\frac{\mu^y e^{-\mu}}{y!}  \\ &= e^{log(\mu^y)+log(e^{-\mu})-log(y!)} \\ &= e^{ylog(\mu)-\mu-log(y!)}\end{aligned}\]

We thus know that \(\theta=log(\mu)\). We can contintue to write the
equation above as follows.

\[=e^{y\theta-e^{\theta}-log(y!)}\]

Thus, we can get:

\[E(X)=\frac{\partial (e^{\theta})}{\partial \theta}=e^{\theta}=\mu\]
\[Var(X)=\frac{\partial^{''} (e^{\theta})}{\partial \theta^2} \phi=\frac{\partial^{''} (e^{\theta})}{\partial \theta^2}=\mu\]

\section{Canonical link}\label{canonical-link}

A link functin can link \(X^T \beta\) to the mean \(\mu\).

That is,

\[g(\mu)=X^T \beta \rightarrow \mu = g^{-1}(X^T \beta)\] We know that

\[\mu = b^{'}(\theta)\]

Thus,

\[ b^{'}(\theta)=g^{-1}(X^T \beta)\] Thus,

\[g=b^{' -1}(\theta)\]

\section{Canonical link - Bernoulli}\label{canonical-link---bernoulli}

PMF of Bernoulli:

\[ \begin{aligned} &= p^y(1-p)^{1-y}  \\ &=e^{log(p^y(1-p)^{1-y})} \\ &= e^{ylog(p)+(1-y)log(1-p)}\\ &= e^{ylog(p)-ylog(1-p)+log(1-p)}\\ &=e^{ylog(\frac{p}{1-p})+log(1-p)} \end{aligned} \]
Copared to the following:

\[f_{\theta}(y)=e^{\frac{y\theta-b(\theta)}{\phi}+c(y,\phi)}\] We need
to change the format of Bernoulli:

\[\theta= log \frac{p}{1-p}\] Thus,
\[e^{\theta}=\frac{p}{1-p} \rightarrow p=\frac{e^{\theta}}{1+e^{\theta}} \]
After that, we can contintue the Bernoulli:

\[\begin{aligned} &= e^{y\theta+log(1-\frac{e^{\theta}}{1+e^{\theta}})} \\ &=e^{y\theta+log(\frac{1}{1+e^{\theta}})} \\ &=e^{y\theta-log(1+e^{\theta})} \end{aligned}\]

Where,

\[b(\theta)=log(1+e^{\theta})\]

We can then try to calculate the derivative:

\[b^{'}(\theta)=\frac{\partial (log(1+e^{\theta}))}{\partial \theta}=\frac{e^{\theta}}{1+e^{\theta}}\]
We know that

\[b^{'}(\theta)=\mu\] Thus, we can get

\[\mu=\frac{e^{\theta}}{1+e^{\theta}}\] We can then calculate the
inverse function:

\[\theta=log(\frac{\mu}{1-\mu})\] Thus,

\[g(\mu)=log(\frac{\mu}{1-\mu})=X^T \beta\]

\section{NR - Bernoulli}\label{nr---bernoulli}

We know that the PMF for Bernoulli:

\[\begin{aligned} &= p^y(1-p)^{1-y} \\  &=e^{y\theta-log(1+e^{\theta})} \\ &=e^{yx^T \beta-log(1+e^{x^T \beta})} \end{aligned}\]
Thus,

\[\ell(\beta|Y,X)=\sum_{i=1}^{n}(Y_iX_i^{T}\beta-log(1+e^{X_i^T \beta}))\]
Thus, teh gradient is:

\[\nabla_{\ell}(\beta)=\sum_{i=1}^{n}(Y_iX_i - \frac{e^{X_i^T \beta}}{1+e^{X_i^T \beta}})\]
The Hessian is:

\[H_{\ell}(\beta)=-\sum_{i=1}^{n}\frac{e^{X_i^T \beta}}{(1+e^{X_i^T \beta})^2}X_iX_i^T\]

Thus,

\[\beta^{k+1}=\beta^k-(H_{\ell}(\beta^k))^{-1}\nabla_{\ell}(\beta^k) \]

\section{Iteratively Re-weighted Least
Squares}\label{iteratively-re-weighted-least-squares}

\chapter{Linear Mixed Models}\label{linear-mixed-models}

\section{LMM}\label{lmm}

The following is a shortened version of Jonathan Rosenblatt's LMM
tutorial. \url{http://www.john-ros.com/Rcourse/lme.html}.

In addition, another reference is from Douglas Bates's R package
document.
\url{https://cran.r-project.org/web/packages/lme4/vignettes/lmer.pdf?fbclid=IwAR1nmmRP9A0BrhKdgBibNjM5acR_spTpXV8QlQGdmTWyQz3ZtV3LYn6kCbQ}

Assume that \(y\) is a function of \(x\) and \(u\), where \(x\) is the
fixed effect and \(u\) is the random effect. Thus, we can get,

\[y|x, u = x'\beta+z'u+\epsilon\]

For random effect, one example can be that you want to test the
treatment effect, and sample 8 observations from 4 groups. You measure
before and after the treatment. In this case, \(x\) represents the
treatment effect, whereas \(z\) represents the group effect (i.e.,
random effect). Note that, in this case, it reminds the paired t-test.
Remember in SPSS, why do we do paired t-test? Typically, it is the case
when we measure a subject (or, participant) twice. In this case, we can
consider each participant as an unit of random effect (rather than as
group in the last example.)

\section{Calculate mean}\label{calculate-mean}

The following code generates 4 numbers (\(N(0,10)\)) for 4 groups. Then,
replicate it within each group.That is, in the end, there are 8
observations.

Note that, in the following code, there are no ``independent
variables''. Both the linear model and mixed model are actually just
trying to calculate the mean. Note that
lmer(y\textasciitilde{}1+1\textbar{}groups) and
lmer(y\textasciitilde{}1\textbar{}groups) will generate the same
results.

\begin{Shaded}
\begin{Highlighting}[]
\KeywordTok{set.seed}\NormalTok{(}\DecValTok{123}\NormalTok{)}
\NormalTok{n.groups <-}\StringTok{ }\DecValTok{4} \CommentTok{# number of groups}
\NormalTok{n.repeats <-}\StringTok{ }\DecValTok{2} \CommentTok{# samples per group}
\CommentTok{#Generating index for observations belong to the same group}
\NormalTok{groups <-}\StringTok{ }\KeywordTok{as.factor}\NormalTok{(}\KeywordTok{rep}\NormalTok{(}\DecValTok{1}\OperatorTok{:}\NormalTok{n.groups, }\DataTypeTok{each=}\NormalTok{n.repeats))}
\NormalTok{n <-}\StringTok{ }\KeywordTok{length}\NormalTok{(groups)}
\CommentTok{#Generating 4 random numbers, assuming normal distribution}
\NormalTok{z0 <-}\StringTok{ }\KeywordTok{rnorm}\NormalTok{(n.groups, }\DecValTok{0}\NormalTok{, }\DecValTok{10}\NormalTok{) }
\NormalTok{z <-}\StringTok{ }\NormalTok{z0[}\KeywordTok{as.numeric}\NormalTok{(groups)] }\CommentTok{# generate and inspect random group effects}
\NormalTok{z}
\end{Highlighting}
\end{Shaded}

\begin{verbatim}
## [1] -5.6047565 -5.6047565 -2.3017749 -2.3017749 15.5870831 15.5870831  0.7050839
## [8]  0.7050839
\end{verbatim}

\begin{Shaded}
\begin{Highlighting}[]
\NormalTok{epsilon <-}\StringTok{ }\KeywordTok{rnorm}\NormalTok{(n,}\DecValTok{0}\NormalTok{,}\DecValTok{1}\NormalTok{) }\CommentTok{# generate measurement error}
\NormalTok{beta0 <-}\StringTok{ }\DecValTok{2} \CommentTok{# this is the actual parameter of interest! The global mean.}
\NormalTok{y <-}\StringTok{ }\NormalTok{beta0 }\OperatorTok{+}\StringTok{ }\NormalTok{z }\OperatorTok{+}\StringTok{ }\NormalTok{epsilon }\CommentTok{# sample from an LMM}

\CommentTok{# fit a linear model assuming independence}
\CommentTok{# i.e., assume that there is no "group things".}
\NormalTok{lm.}\DecValTok{5}\NormalTok{ <-}\StringTok{ }\KeywordTok{lm}\NormalTok{(y}\OperatorTok{~}\DecValTok{1}\NormalTok{)}

\CommentTok{# fit a mixed-model that deals with the group dependence}
\CommentTok{#install.packages("lme4")}
\KeywordTok{library}\NormalTok{(lme4)}
\NormalTok{lme.}\FloatTok{5.}\NormalTok{a <-}\StringTok{ }\KeywordTok{lmer}\NormalTok{(y}\OperatorTok{~}\DecValTok{1}\OperatorTok{+}\DecValTok{1}\OperatorTok{|}\NormalTok{groups) }
\NormalTok{lme.}\FloatTok{5.}\NormalTok{b <-}\StringTok{ }\KeywordTok{lmer}\NormalTok{(y}\OperatorTok{~}\DecValTok{1}\OperatorTok{|}\NormalTok{groups) }
\NormalTok{lm.}\DecValTok{5}
\end{Highlighting}
\end{Shaded}

\begin{verbatim}
## 
## Call:
## lm(formula = y ~ 1)
## 
## Coefficients:
## (Intercept)  
##       4.283
\end{verbatim}

\begin{Shaded}
\begin{Highlighting}[]
\NormalTok{lme.}\FloatTok{5.}\NormalTok{a }
\end{Highlighting}
\end{Shaded}

\begin{verbatim}
## Linear mixed model fit by REML ['lmerMod']
## Formula: y ~ 1 + 1 | groups
## REML criterion at convergence: 36.1666
## Random effects:
##  Groups   Name        Std.Dev.
##  groups   (Intercept) 8.8521  
##  Residual             0.8873  
## Number of obs: 8, groups:  groups, 4
## Fixed Effects:
## (Intercept)  
##       4.283
\end{verbatim}

\begin{Shaded}
\begin{Highlighting}[]
\NormalTok{lme.}\FloatTok{5.}\NormalTok{b }
\end{Highlighting}
\end{Shaded}

\begin{verbatim}
## Linear mixed model fit by REML ['lmerMod']
## Formula: y ~ 1 | groups
## REML criterion at convergence: 36.1666
## Random effects:
##  Groups   Name        Std.Dev.
##  groups   (Intercept) 8.8521  
##  Residual             0.8873  
## Number of obs: 8, groups:  groups, 4
## Fixed Effects:
## (Intercept)  
##       4.283
\end{verbatim}

\section{Test the treatment effect}\label{test-the-treatment-effect}

As we can see that, LLM and paired t-test generate the same t-value.

\begin{Shaded}
\begin{Highlighting}[]
\NormalTok{times<-}\KeywordTok{rep}\NormalTok{(}\KeywordTok{c}\NormalTok{(}\DecValTok{1}\NormalTok{,}\DecValTok{2}\NormalTok{),}\DecValTok{4}\NormalTok{) }\CommentTok{# first time and second time}
\NormalTok{times}
\end{Highlighting}
\end{Shaded}

\begin{verbatim}
## [1] 1 2 1 2 1 2 1 2
\end{verbatim}

\begin{Shaded}
\begin{Highlighting}[]
\NormalTok{data_combined<-}\KeywordTok{cbind}\NormalTok{(y,groups,times)}
\NormalTok{data_combined}
\end{Highlighting}
\end{Shaded}

\begin{verbatim}
##               y groups times
## [1,] -3.4754687      1     1
## [2,] -1.8896915      1     2
## [3,]  0.1591413      2     1
## [4,] -1.5668361      2     2
## [5,] 16.9002303      3     1
## [6,] 17.1414212      3     2
## [7,]  3.9291657      4     1
## [8,]  3.0648977      4     2
\end{verbatim}

\begin{Shaded}
\begin{Highlighting}[]
\NormalTok{lme_diff_times<-}\StringTok{ }\KeywordTok{lmer}\NormalTok{(y}\OperatorTok{~}\NormalTok{times}\OperatorTok{+}\NormalTok{(}\DecValTok{1}\OperatorTok{|}\NormalTok{groups)) }


\NormalTok{t_results<-}\KeywordTok{t.test}\NormalTok{(y}\OperatorTok{~}\NormalTok{times, }\DataTypeTok{paired=}\OtherTok{TRUE}\NormalTok{)}

\NormalTok{lme_diff_times}
\end{Highlighting}
\end{Shaded}

\begin{verbatim}
## Linear mixed model fit by REML ['lmerMod']
## Formula: y ~ times + (1 | groups)
## REML criterion at convergence: 35.0539
## Random effects:
##  Groups   Name        Std.Dev.
##  groups   (Intercept) 8.845   
##  Residual             1.013   
## Number of obs: 8, groups:  groups, 4
## Fixed Effects:
## (Intercept)        times  
##      4.5691      -0.1908
\end{verbatim}

\begin{Shaded}
\begin{Highlighting}[]
\KeywordTok{print}\NormalTok{(}\StringTok{"The following results are from paired t-test"}\NormalTok{)}
\end{Highlighting}
\end{Shaded}

\begin{verbatim}
## [1] "The following results are from paired t-test"
\end{verbatim}

\begin{Shaded}
\begin{Highlighting}[]
\NormalTok{t_results}\OperatorTok{$}\NormalTok{statistic}
\end{Highlighting}
\end{Shaded}

\begin{verbatim}
##         t 
## 0.2664793
\end{verbatim}

\section{Another example}\label{another-example}

\begin{Shaded}
\begin{Highlighting}[]
\KeywordTok{data}\NormalTok{(Dyestuff, }\DataTypeTok{package=}\StringTok{'lme4'}\NormalTok{)}
\KeywordTok{attach}\NormalTok{(Dyestuff)}
\end{Highlighting}
\end{Shaded}

\begin{verbatim}
## The following objects are masked from Dyestuff (pos = 5):
## 
##     Batch, Yield
\end{verbatim}

\begin{Shaded}
\begin{Highlighting}[]
\NormalTok{Dyestuff}
\end{Highlighting}
\end{Shaded}

\begin{verbatim}
##    Batch Yield
## 1      A  1545
## 2      A  1440
## 3      A  1440
## 4      A  1520
## 5      A  1580
## 6      B  1540
## 7      B  1555
## 8      B  1490
## 9      B  1560
## 10     B  1495
## 11     C  1595
## 12     C  1550
## 13     C  1605
## 14     C  1510
## 15     C  1560
## 16     D  1445
## 17     D  1440
## 18     D  1595
## 19     D  1465
## 20     D  1545
## 21     E  1595
## 22     E  1630
## 23     E  1515
## 24     E  1635
## 25     E  1625
## 26     F  1520
## 27     F  1455
## 28     F  1450
## 29     F  1480
## 30     F  1445
\end{verbatim}

\begin{Shaded}
\begin{Highlighting}[]
\NormalTok{lme_batch<-}\StringTok{ }\KeywordTok{lmer}\NormalTok{( Yield }\OperatorTok{~}\StringTok{ }\DecValTok{1} \OperatorTok{+}\StringTok{ }\NormalTok{(}\DecValTok{1}\OperatorTok{|}\NormalTok{Batch)  , Dyestuff )}
\KeywordTok{summary}\NormalTok{(lme_batch)}
\end{Highlighting}
\end{Shaded}

\begin{verbatim}
## Linear mixed model fit by REML ['lmerMod']
## Formula: Yield ~ 1 + (1 | Batch)
##    Data: Dyestuff
## 
## REML criterion at convergence: 319.7
## 
## Scaled residuals: 
##     Min      1Q  Median      3Q     Max 
## -1.4117 -0.7634  0.1418  0.7792  1.8296 
## 
## Random effects:
##  Groups   Name        Variance Std.Dev.
##  Batch    (Intercept) 1764     42.00   
##  Residual             2451     49.51   
## Number of obs: 30, groups:  Batch, 6
## 
## Fixed effects:
##             Estimate Std. Error t value
## (Intercept)  1527.50      19.38    78.8
\end{verbatim}

\section{Full LMM model}\label{full-lmm-model}

In the following, I used the data from the package of lme4. For Days +
(1 \textbar{} Subject), it only has random intercept; in contrast, Days
+ ( Days\textbar{} Subject ) has both random intercept and random slope
for Days. Note that, random effects do not generate specific slopes for
each level of Days, but rather just a variance of all the slopes.

Therefore, we can see that ``Days + ( Days\textbar{} Subject )'' and
``Days + ( 1+Days\textbar{} Subject )'' generate the same results. For
more discussion, you can refer to the following link:
\url{https://www.jaredknowles.com/journal/2013/11/25/getting-started-with-mixed-effect-models-in-r}

\begin{Shaded}
\begin{Highlighting}[]
\KeywordTok{data}\NormalTok{(sleepstudy, }\DataTypeTok{package=}\StringTok{'lme4'}\NormalTok{)}
\KeywordTok{attach}\NormalTok{(sleepstudy)}
\end{Highlighting}
\end{Shaded}

\begin{verbatim}
## The following objects are masked from sleepstudy (pos = 5):
## 
##     Days, Reaction, Subject
\end{verbatim}

\begin{Shaded}
\begin{Highlighting}[]
\NormalTok{fm1 <-}\StringTok{ }\KeywordTok{lmer}\NormalTok{(Reaction }\OperatorTok{~}\StringTok{ }\NormalTok{Days }\OperatorTok{+}\StringTok{ }\NormalTok{(}\DecValTok{1} \OperatorTok{|}\StringTok{ }\NormalTok{Subject), sleepstudy)}
\KeywordTok{summary}\NormalTok{(fm1)}
\end{Highlighting}
\end{Shaded}

\begin{verbatim}
## Linear mixed model fit by REML ['lmerMod']
## Formula: Reaction ~ Days + (1 | Subject)
##    Data: sleepstudy
## 
## REML criterion at convergence: 1786.5
## 
## Scaled residuals: 
##     Min      1Q  Median      3Q     Max 
## -3.2257 -0.5529  0.0109  0.5188  4.2506 
## 
## Random effects:
##  Groups   Name        Variance Std.Dev.
##  Subject  (Intercept) 1378.2   37.12   
##  Residual              960.5   30.99   
## Number of obs: 180, groups:  Subject, 18
## 
## Fixed effects:
##             Estimate Std. Error t value
## (Intercept) 251.4051     9.7467   25.79
## Days         10.4673     0.8042   13.02
## 
## Correlation of Fixed Effects:
##      (Intr)
## Days -0.371
\end{verbatim}

\begin{Shaded}
\begin{Highlighting}[]
\NormalTok{fm2<-}\KeywordTok{lmer}\NormalTok{ ( Reaction }\OperatorTok{~}\StringTok{ }\NormalTok{Days }\OperatorTok{+}\StringTok{ }\NormalTok{( Days}\OperatorTok{|}\StringTok{ }\NormalTok{Subject ) , }\DataTypeTok{data=}\NormalTok{ sleepstudy )}
\KeywordTok{summary}\NormalTok{(fm2)}
\end{Highlighting}
\end{Shaded}

\begin{verbatim}
## Linear mixed model fit by REML ['lmerMod']
## Formula: Reaction ~ Days + (Days | Subject)
##    Data: sleepstudy
## 
## REML criterion at convergence: 1743.6
## 
## Scaled residuals: 
##     Min      1Q  Median      3Q     Max 
## -3.9536 -0.4634  0.0231  0.4633  5.1793 
## 
## Random effects:
##  Groups   Name        Variance Std.Dev. Corr
##  Subject  (Intercept) 611.90   24.737       
##           Days         35.08    5.923   0.07
##  Residual             654.94   25.592       
## Number of obs: 180, groups:  Subject, 18
## 
## Fixed effects:
##             Estimate Std. Error t value
## (Intercept)  251.405      6.824  36.843
## Days          10.467      1.546   6.771
## 
## Correlation of Fixed Effects:
##      (Intr)
## Days -0.138
\end{verbatim}

\begin{Shaded}
\begin{Highlighting}[]
\NormalTok{fm3<-}\KeywordTok{lmer}\NormalTok{ ( Reaction }\OperatorTok{~}\StringTok{ }\NormalTok{Days }\OperatorTok{+}\StringTok{ }\NormalTok{(}\DecValTok{1}\OperatorTok{+}\NormalTok{Days}\OperatorTok{|}\StringTok{ }\NormalTok{Subject ) , }\DataTypeTok{data=}\NormalTok{ sleepstudy )}
\KeywordTok{summary}\NormalTok{(fm3)}
\end{Highlighting}
\end{Shaded}

\begin{verbatim}
## Linear mixed model fit by REML ['lmerMod']
## Formula: Reaction ~ Days + (1 + Days | Subject)
##    Data: sleepstudy
## 
## REML criterion at convergence: 1743.6
## 
## Scaled residuals: 
##     Min      1Q  Median      3Q     Max 
## -3.9536 -0.4634  0.0231  0.4633  5.1793 
## 
## Random effects:
##  Groups   Name        Variance Std.Dev. Corr
##  Subject  (Intercept) 611.90   24.737       
##           Days         35.08    5.923   0.07
##  Residual             654.94   25.592       
## Number of obs: 180, groups:  Subject, 18
## 
## Fixed effects:
##             Estimate Std. Error t value
## (Intercept)  251.405      6.824  36.843
## Days          10.467      1.546   6.771
## 
## Correlation of Fixed Effects:
##      (Intr)
## Days -0.138
\end{verbatim}

\section{Serial correlations in time and
space}\label{serial-correlations-in-time-and-space}

The hierarchical model of \(y|x, u = x'\beta+z'u+\epsilon\) can work
well for correlations within blocks, but not for correlations in time as
the correlations decay in time. The following uses nlme package to
calculate time serial data.

\begin{Shaded}
\begin{Highlighting}[]
\KeywordTok{library}\NormalTok{(nlme)}
\KeywordTok{head}\NormalTok{(nlme}\OperatorTok{::}\NormalTok{Ovary,}\DataTypeTok{n=}\DecValTok{50}\NormalTok{)}
\end{Highlighting}
\end{Shaded}

\begin{verbatim}
## Grouped Data: follicles ~ Time | Mare
##    Mare        Time follicles
## 1     1 -0.13636360        20
## 2     1 -0.09090910        15
## 3     1 -0.04545455        19
## 4     1  0.00000000        16
## 5     1  0.04545455        13
## 6     1  0.09090910        10
## 7     1  0.13636360        12
## 8     1  0.18181820        14
## 9     1  0.22727270        13
## 10    1  0.27272730        20
## 11    1  0.31818180        22
## 12    1  0.36363640        15
## 13    1  0.40909090        18
## 14    1  0.45454550        17
## 15    1  0.50000000        14
## 16    1  0.54545450        18
## 17    1  0.59090910        14
## 18    1  0.63636360        16
## 19    1  0.68181820        17
## 20    1  0.72727270        18
## 21    1  0.77272730        18
## 22    1  0.81818180        17
## 23    1  0.86363640        14
## 24    1  0.90909090        12
## 25    1  0.95454550        12
## 26    1  1.00000000        14
## 27    1  1.04545500        10
## 28    1  1.09090900        11
## 29    1  1.13636400        16
## 30    2 -0.15000000         6
## 31    2 -0.10000000         6
## 32    2 -0.05000000         8
## 33    2  0.00000000         7
## 34    2  0.05000000        16
## 35    2  0.10000000        10
## 36    2  0.15000000        13
## 37    2  0.20000000         9
## 38    2  0.25000000         7
## 39    2  0.30000000         6
## 40    2  0.35000000         8
## 41    2  0.40000000         8
## 42    2  0.45000000         6
## 43    2  0.50000000         8
## 44    2  0.55000000         7
## 45    2  0.60000000         9
## 46    2  0.65000000         6
## 47    2  0.70000000         4
## 48    2  0.75000000         5
## 49    2  0.80000000         8
## 50    2  0.85000000        11
\end{verbatim}

\begin{Shaded}
\begin{Highlighting}[]
\NormalTok{fm1Ovar.lme <-}\StringTok{ }\NormalTok{nlme}\OperatorTok{::}\KeywordTok{lme}\NormalTok{(}\DataTypeTok{fixed=}\NormalTok{follicles }\OperatorTok{~}\StringTok{ }\KeywordTok{sin}\NormalTok{(}\DecValTok{2}\OperatorTok{*}\NormalTok{pi}\OperatorTok{*}\NormalTok{Time) }\OperatorTok{+}\StringTok{ }\KeywordTok{cos}\NormalTok{(}\DecValTok{2}\OperatorTok{*}\NormalTok{pi}\OperatorTok{*}\NormalTok{Time), }
                   \DataTypeTok{data =}\NormalTok{ Ovary, }
                   \DataTypeTok{random =} \KeywordTok{pdDiag}\NormalTok{(}\OperatorTok{~}\KeywordTok{sin}\NormalTok{(}\DecValTok{2}\OperatorTok{*}\NormalTok{pi}\OperatorTok{*}\NormalTok{Time)), }
                   \DataTypeTok{correlation=}\KeywordTok{corAR1}\NormalTok{() )}
\KeywordTok{summary}\NormalTok{(fm1Ovar.lme)}
\end{Highlighting}
\end{Shaded}

\begin{verbatim}
## Linear mixed-effects model fit by REML
##  Data: Ovary 
##        AIC     BIC   logLik
##   1563.448 1589.49 -774.724
## 
## Random effects:
##  Formula: ~sin(2 * pi * Time) | Mare
##  Structure: Diagonal
##         (Intercept) sin(2 * pi * Time) Residual
## StdDev:    2.858385           1.257977 3.507053
## 
## Correlation Structure: AR(1)
##  Formula: ~1 | Mare 
##  Parameter estimate(s):
##       Phi 
## 0.5721866 
## Fixed effects: follicles ~ sin(2 * pi * Time) + cos(2 * pi * Time) 
##                        Value Std.Error  DF   t-value p-value
## (Intercept)        12.188089 0.9436602 295 12.915760  0.0000
## sin(2 * pi * Time) -2.985297 0.6055968 295 -4.929513  0.0000
## cos(2 * pi * Time) -0.877762 0.4777821 295 -1.837159  0.0672
##  Correlation: 
##                    (Intr) s(*p*T
## sin(2 * pi * Time)  0.000       
## cos(2 * pi * Time) -0.123  0.000
## 
## Standardized Within-Group Residuals:
##         Min          Q1         Med          Q3         Max 
## -2.34910093 -0.58969626 -0.04577893  0.52931186  3.37167486 
## 
## Number of Observations: 308
## Number of Groups: 11
\end{verbatim}

\chapter{Generalized Linear Mixed
Models}\label{generalized-linear-mixed-models}

\section{Basics of GLMM}\label{basics-of-glmm}

Recall the formula in the probit model:

\[Y^*=X\beta+\epsilon, \epsilon \sim N(0,\sigma^2)=N(0,I)\] Similar to
LMM, binary model with random effect can be written as follows.

\[Y^*=X\beta+ Z u+\epsilon\] where,

\[\epsilon \sim N(0,I)\] \[u \sim N(0, D)\]

We also assume \(\epsilon\) and \(u\) are independent.Thus, we know that
\(D\) represents the virances of the random effects. If we make
\(u =1\), the model becomes the usual probit model. McCulloch (1994)
states that there are a few advantages to use probit, rather than logit
models. (Note that, however, probit is not canonical link function, but
logit is!)

The following is the note from Charle E. McCulloch's ``Maximum
likelihood algorithems for Generalized Linear Mixed Models''

\section{Some References}\label{some-references}

\url{http://www.biostat.umn.edu/~baolin/teaching/linmods/glmm.html}

\url{http://www.biostat.umn.edu/~baolin/teaching/probmods/GLMM_mcmc.html}

\url{https://bbolker.github.io/mixedmodels-misc/glmmFAQ.html}

\bibliography{book.bib,packages.bib}

\end{document}
